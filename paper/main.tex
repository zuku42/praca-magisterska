% Style for a MSc paper at Warsaw School of Economics
% Michał Ramsza
% Friday, December 14, 2012

% --- document class and other global stuff ---------------------------
\documentclass[english, twoside, 12pt, a4paper]{article}

% --- packages --------------------------------------------------------
\usepackage{textcomp}
\usepackage{times}
\usepackage{amsmath}
\usepackage{amsfonts}
\usepackage{amssymb}
\usepackage{amsthm}
\usepackage[T1]{fontenc}
\usepackage[utf8]{inputenc}
\usepackage{graphicx}
\usepackage{xcolor}
\usepackage{enumitem}
\usepackage[english]{babel}
\usepackage[centering, left=3.5cm, right=2.5cm, textheight=24cm]{geometry}

% --- packages for citations ------------------------------------------
\usepackage{natbib}
\AtBeginDocument{\renewcommand{\harvardand}{and}}

% --- package for automatic insertion of R code -----------------------
\usepackage{listings}
\lstset{language=R,%
   numbers=left,%
   tabsize=3,%
   numberstyle=\footnotesize,%
   basicstyle=\ttfamily \footnotesize \color{black},%
   escapeinside={(*@}{@*)}}

% --- support for links -----------------------------------------------	
\usepackage{url}
\usepackage{hyperref}
\hypersetup{colorlinks=true,
            linkcolor=black,
            citecolor=darkgray,
            urlcolor=darkgray,
            pagecolor=darkgray}

% --- support for large tables and other stuff ------------------------	
\usepackage{longtable}
%\usepackage{subfigure} % this package will not work with subcaption package
\usepackage{float}
\usepackage{caption}
\usepackage{subcaption}
\usepackage{wrapfig}
\usepackage{pdflscape} % relevant for wide tables (rotating pages)

% --- support for game theory ------------------------------------------
\usepackage{sgame}

% --- support for no widows --------------------------------------------
\usepackage[defaultlines=4,all]{nowidow}

% --- definitions for environments -------------------------------------
\theoremstyle{definition}
    \newtheorem{condition}{Assumption}
    \newtheorem{example}{Example}      

\theoremstyle{plain}
    \newtheorem{definition}{Definition}    
    \newtheorem{proposition}{Proposition}
    \newtheorem{theorem}{Theorem}
    \newtheorem{cor}{Corollary}

\theoremstyle{remark}
    \newtheorem{remark}{Remark}

% --- other settings --------------------------------------------------
\linespread{1.5}
\frenchspacing
\sloppy
\allowdisplaybreaks[4]
\raggedbottom
\clubpenalty=10000
\widowpenalty=10000

% --- only if required ------------------------------------------------
\AtBeginDocument{\renewcommand*{\figurename}{Figure}}
\AtBeginDocument{\renewcommand*{\tablename}{Table}}

% --- changing definition of footnote ---------------------------------
\makeatletter
\renewcommand\footnotesize{%
   \@setfontsize\footnotesize\@ixpt{10}%
   \abovedisplayskip 8\p@ \@plus2\p@ \@minus4\p@
   \abovedisplayshortskip \z@ \@plus\p@
   \belowdisplayshortskip 4\p@ \@plus2\p@ \@minus2\p@
   \def\@listi{\leftmargin\leftmargini
               \topsep 4\p@ \@plus2\p@ \@minus2\p@
               \parsep 2\p@ \@plus\p@ \@minus\p@
               \itemsep \parsep}%
   \belowdisplayskip \abovedisplayskip
}
\makeatother


% ---------------------------------------------------------------------
\begin{document}

% --- strona tytulowa -------------------------------------------------
\begin{titlepage}
\centering

\includegraphics[width=0.66\textwidth]{logo.JPG}

\vspace*{0.5cm}
Studium <licencjackie/magisterskie>\\
\begin{flushleft}
Kierunek: <wpisać kierunek>\\
%Specjalność: <specjalność> % w przypadku braku należy pominać
%Forma studiów: <forma studiów (stacjonarne, itd.)>
\end{flushleft}

\vspace*{.5cm}
\rule{0cm}{1cm}\hfill
\begin{minipage}{9cm}
Imie i nazwisko autora: <imię nazwisko>\\
Nr albumu: <12345>
\end{minipage}

\vspace*{1cm}
\begin{minipage}{12cm}
\centering
\Large
\textbf{<tytuł>}
\end{minipage}

\vspace*{2cm}
\rule{0cm}{1cm}\hfill
\begin{minipage}{9cm}
Praca licencjacka napisana\\
w Katedrze Matematyki i Ekonomii Matematycznej\\
pod kierunkiem naukowym\\
dr hab. Michała Ramszy
\end{minipage}

\vfill
Warszawa <rok>
\end{titlepage}

\rule{1ex}{0ex}\clearpage

% --- table of contents -----------------------------------------------
\cleardoublepage
\tableofcontents

% --- chapter ---------------------------------------------------------
\cleardoublepage
\section{Introduction}

This template is for the BSc papers at Warsaw School of Economics. 

% --- chapter ---------------------------------------------------------
\clearpage
\section{Basic things}

\subsection{Compiling \LaTeX files}

The \verb+.tex+ file is just a plain text file. It contains the \LaTeX formatting codes together with the content of a paper. To get a \verb+.pdf+ file you have to compile the \verb+.tex+ file using a sequence \verb+pdflatex+, \verb+biblatex+, \verb+pdflatex+, \verb+pdflatex+. This sequence is a default in most editors designed for use with \LaTeX.

\subsection{Basic formatting for a text}

Paragraphs are coded by an empty line. That is is you want to start a new paragraph it is enough to leave an empty line and start typing like that:
\begin{verbatim}
This is the first paragraph.

This is the next paragraph.
\end{verbatim}

Everything about the paragraph is formatted for you including all indents and spacings. Again, you don't have to take care of it manually.

Basic text formatting, e.g. bold face and italic, is achieved with the following commands: \verb+\textbf{}+, \verb+\textit{}+, \verb+\underline{}+, producing \textbf{text}, \textit{text}, \underline{text}. I suggest not overusing those commands!

Alignment is done through environments \verb+center+, \verb+flushleft+ and \verb+\flushright+ giving the following examples.

\begin{center}
  This is centered.
\end{center}

\begin{flushleft}
  This is aligned to the left.
\end{flushleft}

\begin{flushright}
  This is aligned to the right. 
\end{flushright}

In other environments it is possible to use \verb+\centering+ to center content of that environment (like in \verb+figure+ or \verb+table+ environments).

\subsection{Fonts and fonts' sizes}

You do not change fonts and fonts' sizes! Technically it can be done but I will reject this.

% --- chapter ---------------------------------------------------------
\clearpage
\section{Mathematics}

This is testing footnotes\footnote{This is a footnote. We can put some math here \( x^2 - f(x) = g(x^2) \) which is not encouraged but sometimes necessary. The other thing we can do is to put here an URL \url{https://tex.stackexchange.com/questions/249415/set-font-size-for-footnotes}. }.

\subsection{Basic mathematics}

There are two types of mathematics inside a \LaTeX{} document. The first one is the in-line mathematics and the displayed mathematics. The first one looks like this: \( F(x) = \int_{-\infty}^{x} f(\omega) d\omega \) with the code looking like this: \verb!\( F(x) = \int_{-\infty}^{x} f(\omega) d\omega \)!. The displayed mathematics looks like that
\[
F(x) = \int_{-\infty}^{x} f(\omega) d\omega
\]
with the code
\begin{verbatim}
\[
F(x) = \int_{-\infty}^{x} f(\omega) d\omega
\]
\end{verbatim}
As you can see the same code is formatted differently depending on the type of mathematics.

\subsection{Referencing mathematics and other things}

To reference mathematics (only displayed formulas) you use the \verb+equation+ environment with a \verb+\label{}+ within. The reference is done through the \verb+\ref{}+ command. The example is
\begin{equation}
\label{eq:this-is-very-important-equation}
F(x) = \int_{-\infty}^{x} f(\omega) d\omega.
\end{equation}
To reference the equation you use the \verb+\ref{}+ command giving (\ref{eq:this-is-very-important-equation}). The \verb+\label{}+ / \verb+\ref{}+ pair works for anything that can be referenced.

\subsection{Some more mathematical formulas}

Here are slightly more complex formulas. Let \( A  \) be a matrix
\[
A =
\left(
\begin{bmatrix}
1                   & \alpha^2       \\
2                   & \sqrt{\pi} - \log(x-\sin(y))
\end{bmatrix}^{2}
- 
\begin{bmatrix}
1                   & f(x)           \\
2                   & g(y)
\end{bmatrix}
\cdot
\begin{bmatrix}
x                                    \\
y
\end{bmatrix}
\right),
\]
where
\[
f(x) = 
\left\{
  \begin{aligned}
    \frac{1}{x}     & \quad \text{for \(x<-\frac{1}{2}\),} \\
    \frac{1}{1+x^2} & \quad \text{for \(x \geq -\frac{1}{2}\)}
  \end{aligned}
\right.
\]
and
\[
g(y) = \sin\left(\frac{\mathrm{\mathbf{E}}(X)}{\cos(y) + \log(y)}\right), 
\quad\text{where \( X \sim \mathrm{N}(0, \sigma)  \).}
\]

It is very easy to typeset a normal form game. Below is an example of such a game. 

\begin{game}{3}{3}
    & $L$    & $M$    & $H$    \\
$L$ & $16,9$ & $3,13$ & $0,3$  \\
$M$ & $21,1$ & $10,4$ & $-1,0$ \\
$H$ & $9,0$  & $5,-4$ & $-5,-15$
\end{game}

% --- chapter ---------------------------------------------------------
\clearpage
\section{Figures and tables}

Both figures and tables use the same ideas. To insert a table you use the \verb+table+ environment. This is an example of a simple table.

\begin{table}[hbt]
  \centering

  \captionsetup{margin=10pt,font=small,labelfont=bf,width=.8\textwidth}

  \caption[Short name for a table]{This is an example of a table.}
  \label{tab:exceptional-table}

\vspace*{2ex}

  \begin{tabular}{lccc}
    Name        & property 1 & property 2 & property 3 \\ \hline
    Michael     & 23         & 34         & --         \\
    John        & 34         & --         & 28         \\
    Mr. Niceguy & 123        & 231        & 312        \\ \hline
  \end{tabular}
\end{table}

Table~\ref{tab:exceptional-table} is a very simple table and much more is possible.

To insert a figure you need to have a figure. In the catalog there are two figures and the following is an example of the \verb+figure+ environment.

\begin{figure}[hbt]
  \centering

  \begin{subfigure}[t]{0.45\textwidth}
    \includegraphics[width=\textwidth]{./figure-1}
  \end{subfigure}

  \captionsetup{margin=10pt,font=small,labelfont=bf,width=.8\textwidth}

  \caption[Short name]{This is just an example. \textit{Source:} own calculations.}\label{fig:xxx1}
\end{figure}

\begin{figure}[hbt]
  \centering
  \begin{subfigure}[t]{0.45\textwidth}
    \includegraphics[width=\textwidth]{./figure-1}
    \caption{This is a caption for the first figure. This caption is wrapped at the right width and the hight is being compensated.}
    \label{fig:xxxa}
  \end{subfigure}
  \hfill
  \begin{subfigure}[t]{0.45\textwidth}
    \includegraphics[width=\textwidth]{figure-2}
    \caption{This is another caption.}
    \label{fig:xxxb}
  \end{subfigure}
  
  \captionsetup{margin=10pt,font=small,labelfont=bf,width=.8\textwidth}

  \caption[Short caption 2]{This is the main caption and it is below the figures. \textit{Source:} own calculations}\label{fig:xxx}
\end{figure}

Figure~\ref{fig:xxx} is a slightly more complex than just a simple figure but it is useful to have such template. It is possible to refrence subfigures as \ref{fig:xxxa} and \ref{fig:xxxb}.

% --- chapter ---------------------------------------------------------
\clearpage
\section{Bibliography}

\begin{wrapfigure}{r}{.5\textwidth}
\centering

\includegraphics[width=.4\textwidth]{figure-2}

\captionsetup{margin=10pt,font=small,labelfont=bf,width=.42\textwidth}

  \caption[Short caption 2]{This is how one can wrap a text around a figure. \textit{Source:} own calculations}\label{fig:yyy}


\end{wrapfigure}

The content for the bibliography is in a different file named \verb+refs.bib+. You can change the name but then you have to change the information in this file from \verb+\bibliography{refs}+ to \verb+\bibliography{new-name}+ where \verb+new-name+ is the name of your file. The file \verb+refs.bib+ contains some examples for books and papers.

The process of citation is simple. The command  \verb+\cite{garland2010}+ gives this \cite{garland2010} and puts all information into the bibliography section  at the end. Everything is sorted and formatted for you so that you don't have to worry about this. An example of a paper with many authors is \cite{benaim2003} or \cite{osborne1998}. 

\begin{longtable}{rrrrr}
\caption{Binary variables used in the VAR model}\label{tab:1}     \\
  \hline
 t   & year & elections & crises & tax cuts                       \\ 
  \hline
  \endfirsthead
  \multicolumn{5}{c}%
{\tablename\ \thetable\ -- \textit{Continued from previous page}} \\
\hline
t    & year & elections & crises & tax cuts                       \\ 
\hline
\endhead
\hline \multicolumn{5}{r}{\textit{Continued on next page}}        \\
\endfoot
\hline
\endlastfoot
  1  & 1961 & 0         & 0      & 0                              \\ 
  2  & 1962 & 0         & 0      & 0                              \\ 
  3  & 1963 & 0         & 0      & 0                              \\ 
  4  & 1964 & 1         & 0      & 0                              \\ 
  5  & 1965 & 0         & 0      & 1                              \\ 
  6  & 1966 & 0         & 0      & 0                              \\ 
  7  & 1967 & 0         & 0      & 0                              \\ 
  8  & 1968 & 1         & 0      & 0                              \\ 
  9  & 1969 & 0         & 0      & 0                              \\ 
  10 & 1970 & 0         & 0      & 0                              \\ 
  11 & 1971 & 0         & 0      & 0                              \\ 
  12 & 1972 & 1         & 0      & 0                              \\ 
  13 & 1973 & 0         & 0      & 0                              \\ 
  14 & 1974 & 0         & 1      & 0                              \\ 
  15 & 1975 & 0         & 1      & 0                              \\ 
  16 & 1976 & 1         & 0      & 0                              \\ 
  17 & 1977 & 0         & 0      & 0                              \\ 
  18 & 1978 & 0         & 0      & 0                              \\ 
  19 & 1979 & 0         & 0      & 0                              \\ 
  20 & 1980 & 1         & 0      & 0                              \\ 
  21 & 1981 & 0         & 0      & 0                              \\ 
  22 & 1982 & 0         & 1      & 1                              \\ 
  23 & 1983 & 0         & 0      & 0                              \\ 
  24 & 1984 & 1         & 0      & 0                              \\ 
  25 & 1985 & 0         & 0      & 0                              \\ 
  26 & 1986 & 0         & 0      & 1                              \\ 
  27 & 1987 & 0         & 0      & 0                              \\ 
  28 & 1988 & 1         & 0      & 0                              \\ 
  29 & 1989 & 0         & 0      & 0                              \\ 
  30 & 1990 & 0         & 0      & 0                              \\ 
  31 & 1991 & 0         & 1      & 0                              \\ 
  32 & 1992 & 1         & 0      & 0                              \\ 
  33 & 1993 & 0         & 0      & 0                              \\ 
  34 & 1994 & 0         & 0      & 0                              \\ 
  35 & 1995 & 0         & 0      & 0                              \\ 
  36 & 1996 & 1         & 0      & 0                              \\ 
  37 & 1997 & 0         & 0      & 0                              \\ 
  38 & 1998 & 0         & 0      & 0                              \\ 
  39 & 1999 & 0         & 0      & 0                              \\ 
  40 & 2000 & 1         & 0      & 0                              \\ 
  41 & 2001 & 0         & 1      & 1                              \\ 
  42 & 2002 & 0         & 0      & 1                              \\ 
  43 & 2003 & 0         & 0      & 1                              \\ 
  44 & 2004 & 1         & 0      & 0                              \\ 
  45 & 2005 & 0         & 0      & 0                              \\ 
  46 & 2006 & 0         & 0      & 0                              \\ 
  47 & 2007 & 0         & 0      & 0                              \\ 
  48 & 2008 & 1         & 1      & 0                              \\ 
  49 & 2009 & 0         & 1      & 1                              \\ 
  50 & 2010 & 0         & 0      & 1                              \\ 
  51 & 2011 & 0         & 0      & 0                              \\ 
  52 & 2012 & 1         & 0      & 0                              \\ 
  53 & 2013 & 0         & 0      & 0                              \\ 
  54 & 2014 & 0         & 0      & 0                              \\ 
  55 & 2015 & 0         & 0      & 0                              \\ 
   \hline
\end{longtable}




% --- appendices ------------------------------------------------------
\appendix

% ---------------------------------------------------------------------
\clearpage
\section{Appendix: Some important stuff}

This appendix contains all the necessary important stuff, blah, blah, blah ...

\begin{landscape}
{\footnotesize
\begin{longtable}{lll}
\caption{Tutaj jest tytuł tablicy}\label{tab:nowatablica1}\\
\hline
Nazwa atrybutu & Wartości & Opis \\ 
\hline
\endfirsthead
\multicolumn{3}{c}%
{\tablename\ \thetable\ -- \textit{kontynuacja z poprzedniej strony}} \\
\hline
Nazwa atrybutu & Wartości & Opis \\
\hline
\endhead
\hline \multicolumn{3}{r}{\textit{kontynuowane na następnej stronie}} \\
\endfoot
\hline
\endlastfoot
chk\_acct & - & stan środków na rachunku bieżącym (jakościowa)\\ 
 & A11 & ... \textless 0 Marek Niemieckich\\  
 & A12 & 0 \textless ... \textless 200 Marek Niemieckich\\  
 & A13 & ... \textgreater 200 Marek Niemieckich\\  
 & A14 & brak rachunku bieżącego\\  
duration & - & czas trwania kredytu w miesiącach (numeryczna)\\  
history & - & przeszłość kredytowa (jakościowa)\\  
 & A30 & brak kredytów w historii/wszystkie kredyty poprawnie spłacone\\  
 & A31 & wszystkie kredyty poprawnie spłacone (zaciągnięte w tym banku)\\  
 & A32 & kredyty poprawnie spłacane po dzień dzisiejszy\\  
 & A33 & opóźnienia w poprzednich spłatach kredytu\\  
 & A34 & konto krytyczne/zaciągnięte kredyty w innych bankach\\  
purpose & - & cel (jakościowa)\\  
 & A40 & nowy samochód\\  
 & A41 & używany samochód\\  
 & A42 & meble\\  
 & A43 & telewizor\\  
 & A44 & urządzenia gospodarstwa domowego\\  
 & A45 & remont\\  
 & A46 & edukacja\\  
 & A47 & wakacje\\  
 & A48 & przekwalifikowanie\\  
 & A49 & biznes\\  
 & A410 & inne\\  
amount & - & kwota kredytu (numeryczna)\\  
say\_acct & - & saldo na rachunku oszczędnościowym/wartość posiadanych obligacji (jakościowa)\\  
 & A61 & ... \textless100 Marek Niemieckich\\  
 & A62 & 100 \textless= ... \textless 500 Marek Niemieckich\\  
 & A63 & 500 \textless= ... \textless 1000 Marek Niemieckich\\  
 & A64 & ... \textgreater= 1000 Marek Niemieckich\\  
 & A65 & nieznane/ brak oszczędności\\  
employment & - & czas zatrudnienia w obecnej pracy (jakościowa)\\  
 & A71 & brak zatrudnienia\\  
 & A72 & ... \textless 1 rok\\  
 & A73 & 1 \textless= ... \textless 4 lata\\  
 & A74 & 4 \textless= ... \textless 7 lat\\  
 & A75 & ... \textgreater= 7 lat\\  
install\_rate & - & wielkość raty jako procent rozporządzalnego przychodu (liczbowa)\\  
pstatus & - & płeć i stan cywilny (jakościowa)\\  
 & A91 & mężczyzna; rozwodnik/w separacji\\  
 & A92 & kobieta; rozwiedziona/ w separacji/ mężatka\\  
 & A93 & mężczyzna ; wolny\\  
 & A94 & mężczyzna ; żonaty/ wdowiec\\  
 & A95 & kobieta ; wolna\\  
other\_debtor & - & inni dłużnicy/ poręczyciele (jakościowa)\\  
 & A101 & brak\\  
 & A102 & współkredytobiorca\\  
 & A103 & poręczyciel\\  
property & - & własność/ mienie (jakościowa)\\  
 & A121 & nieruchomość\\  
 & A122 & (jeśli nie A121) umowa oszczędnościowa/ ubezpieczenie na życie\\  
 & A123 & (jeśli nie A121/A122) samochód lub inne\\  
 & A124 & nieznane\\  
timer\_resid & - & czas zamieszkania w aktualnym miejscu zamieszkania (liczbowa)\\  
age & - & wiek w latach (liczbowa)\\  
other\_install & - & inne zobowiązania ratalne (jakościowa)\\  
 & A141 & bank\\  
 & A142 & sklepy\\  
 & A143 & brak\\  
housing & - & warunki mieszkaniowe (jakościowa)\\  
 & A151 & wynajem\\  
 & A152 & własność\\  
 & A153 & zamieszkanie bez ponoszenia kosztów\\  
other\_credits & - & liczba aktualnych kredytów w tym banku (liczbowa)\\  
job & - & praca (jakościowa)\\  
 & A171 & bezrobotny/niewykwalifikowany; cudzoziemiec\\  
 & A172 & niewykwalifikowany; rezydent\\  
 & A173 & wykwalifikowany pracownik/urzędnik\\  
 & A174 & menadżer/ samozatrudniony/ wysocewykwalifikowany/ wyższy urzędnik\\  
num\_depend & - & liczba osób na utrzymaniu (liczbowa)\\  
telephone & - & telefon (jakościowa)\\  
 & A191 & brak\\  
 & A192 & tak, zarejestrowany pod nazwiskiem klienta\\  
foreign & - & pracownik zagraniczny (jakościowa)\\  
 & A201 & tak\\  
 & A202 & nie\\  
response & - & decyzja kredytowa\\  
 & 1 & tak\\  
 & 2 & nie\\ 
 \hline
\end{longtable}}
\end{landscape}



% --- bibliography ----------------------------------------------------
\clearpage
\bibliographystyle{agsm}
\bibliography{refs}

% --- abstract --------------------------------------------------------
\clearpage
\addcontentsline{toc}{section}{List of tables}
\listoftables

% --- abstract --------------------------------------------------------
\clearpage
\addcontentsline{toc}{section}{List of figures}
\listoffigures



% --- abstract --------------------------------------------------------
\clearpage
\addcontentsline{toc}{section}{Streszczenie}
\section*{Streszczenie}

Tutaj zamieszczają Państwo streszczenie pracy. Streszczenie powinno być długości około pół strony.


\end{document}


%%% Local Variables:
%%% mode: latex
%%% TeX-master: t
%%% End:
